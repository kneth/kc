\documentstyle{article}
\author{Kenneth Geisshirt}
\title{The {\tt kin} precompilers}
\date{4 August 1992}
\begin{document}
\maketitle
\section{Preface}
The {\tt kin} programs are precompilers, which translate chemical equations 
into runnable programs. The generated program 'solves' the chemical system.

The precompilers are written in ANSI-C, so porting them are easy (recompile them).
The output is fairly difficult, because it is HP-Pascal code (difficult for people, who
don't have a HP-Pascal compiler on their system). 

The precompiler transforms the chemical model into differential equations using the
mass-action law. The system consists of two programs. The first one is {\tt kinc}
which generates velocities and jacobians. The other is {\tt kins} which does
the same plus generates the second order tensor. 

The integration finds the concentrations with respect of time.

A older version of the system is running as an interactive PC-program (rather,
running on a Partner from Regnecentralen). This version can also do graphics 
manipulations on the data obtained from the integration process.

\section{How to run it}
The requirements to the computer, which is going to run {\tt kin} is, that it 
has a C compiler (ANSI compatible) and a HP-Pascal. Assuming that the computer 
does not have a HP-Pascal, then the requirements are a C-compiler and a  
Pascal-to-C translator, e. g. {\em p2c} by Daves Gillespie \footnote{The system
installed (and running) on {\em tiger} at PICHT is of the later type.}. In the following
it is assumed that no Pascal compiler is present.

Running the {\tt kin} system is a bit tricking at the moment, because I don't know
the Unix operating system well enough. But the following procedure can be used. 
\begin{enumerate}
 \item You have to be in the right directory. In this context the right directory
       is up to you, but you have to write it into the {\tt tkin} script (variable
       {\tt MYPATH} contains the information).
       The directory has to be a subdirectory to the one containing the integrator and
       other vital files. The script doesn't have to be in either directories.    
 \item The system is now executed by the command {\tt tkin model}, where {\tt model} is
       file containing your model. The integrator will be started as soon as the compilation
       is done, so redirection is recommended.
\end{enumerate}

\section{Minimal user's manual}
This section is not intented to be a full description of the system, but a brief introduction
so the reader should be able to use the system.

The input format is straightforward (when you have tried it a few times). The model
consists of chemical equations, initial concentrations and linear constrains. The last
two sections are optional. Parameters can be defined every where in the 'source'
file.

For further information on the input, I have written some small examples. These shows some of the
key points in the {\tt kin} system.

\subsection{Equations}
All the equations are on the form:
\begin{verbatim}
  number : A + B + ... -> Q + R + ... ; k> = number
\end{verbatim}
The first number is a positive integer, while the last number is a floating-point constant.
The colon after the first number have to be there, otherwise it is a syntax error. The reactants
and products are written in the usual way. Coefficients are naturally allowed. The names of the
species do not have anything to do with the real world; they are 'just' variables. Two kinds of
reactions are allowed. They are one-way reaction {\tt ->} and two-ways reaction {\tt <=>}. For two-
way reactions two rate constants must be supplied. The rate constants are written on the same line after a
semicolon. A {\tt >} denoted the reaction from left to right, while {\tt <} denotes the reaction from
righ to left. A space after {\tt >} and {\tt <} is very important.

\subsection{Initial concentrations}
Initial concentrations are written after the equations. The default value is $0.0$, but the user can
request other values at $t=0$. The syntax is simple; it is:
\begin{verbatim}
  [J](0) = number
\end{verbatim}
where {\tt J} is a species, and {\tt number} is a constant (can of course be a parameter). 

\subsection{Linear constrains}
The linear constrains in this version of {\tt kin} are very simple. As the name says, it is only
linear, and a constrain has the form:
\begin{verbatim}
  [J] = number - [A] - ...
\end{verbatim}
Of course there can be coefficients before a concentration, but they have to be numbers. 
Further, the concentrations on the right side can be omitted, i. e.
a concentration is constant at all time.

\subsection{Parameters}
A parameter declaration has the form:
\begin{verbatim}
  name = expr
\end{verbatim}
where {\tt name} is the symbolic name of the parameter, and {\tt expr} is a expression. The expression
is a general expression, i. e. the four mathematical operators can be used. There are some predefined
parameters. They are shown below.

\begin{tabular}{llr}
 Name   & Description    & Default value \\ 
 stime  & start time     & $0$ \\ 
 dtime  & step length    & $1$ \\
 etime  & end time       & $10$ \\
 epsr   & relative error & $10^{-3}$ \\
 epsa   & absolute error & $10^{-20}$ \\
 name   & name of output file & "kinwrk.dat" 
\end{tabular}

The value of these parameters will be copied into the integrator, they and will therefore have
and an effect on the program.

\subsection{Call of external functions}
The {\tt kin} system has a feature of calling external functions. The only 
predefined function is {\tt arrh} which computes rate constants according
to the Arrhenius' formula. The form of a call of an external function is:
\begin{verbatim}
 func <: p1, p2, ..., pn :>(0)
\end{verbatim}
where {\tt func} is the name of the function and {\tt pi} is the parameters
to the function. The result of the call is returned in the {\em first}
parameter, i. e. a call to the Arrhenius function will be:
\begin{verbatim}
  arrh <: k, a, E, T :>(0)
\end{verbatim}

An example of the format of external functions can be found in the file
{\tt arrh.c}, which is supplied with the {\tt kin} system.

The use of external functions will cause a warning from {\tt p2c}, because
a parameter is not defined {\footnote{The warning is: {\tt Expected a 
expression, found a semicolon}}}, but this warning can be ignored. 

\subsection{Comments}
Comments can be placed everywhere in the source file. The comments are in C-
style, i. e. they have the form:

\begin{verbatim}
  /* This is a comment */
\end{verbatim}

\subsection{Numbers}
There is a minor problem when using the system on {\em tiger}. The   
{\em p2c} translator seems to have problem when translating double
precision floating-point numbers, e. g. {\tt 1.0L-1}. Don't use
them but use only normal numbers, e. g. {\tt 1.0E-1}.

\subsection{Limits}
{\tt kin} has some limitations, especially in the number of reactions and species. The table below shows
these limitations.
\begin{tabular}{lr}
 Reactions  & 150 \\  
 Species    &  50 \\
 Reactants  &   5 \\
 Constrains &   5 \\
 Species in constrains & 10 \\
 Parameters & 100 \\
 External function & 50 
\end{tabular}

\section{Good and bad things}
The {\tt kin} has its advantages and its disadvantages. In this section I will shortly discuss some
of them.

The disadvantages are:
\begin{enumerate}
 \item Only mass-action law.
 \item No ions and therefore no diffusion can be modeled.
 \item The syntax of the constrains is not smart, and they can only be linear.
 \item No equilibriums can be modeled directly, but only as fast two-way reactions.
 \item Code generator is not easy to port (have to use strange translators).
\end{enumerate}

The advantages are:
\begin{enumerate}
 \item Easy syntax for chemical equations. It is must like ordinary mechanism. 
 \item Does the boring job, i. e. the chemist can do what is his job: making
       chemical models \footnote{Boring work often leads to errors!}.
\end{enumerate}
\end{document}
