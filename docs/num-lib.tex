% Last updated: 5 Oct 1994
\documentclass[12pt]{article}

\newcommand{\ie}[0]{{\it i.e.\ \/}}
\newcommand{\eg}[0]{{\it e.g.\ \/}}
\newcommand{\diff}[2]{\frac{{\mathrm d}#1}{{\mathrm d}#2}}

\setlength{\parindent}{0cm}
\setlength{\parskip}{0.1cm}

\title{A Small Package of \\ Numerical Methods in ANSI C}
\author{Kenneth Geisshirt \and Keld Nielsen}
\date{5 October 1994}
\begin{document}

\maketitle
\tableofcontents

\section{Introduction}
\label{sec:Intro}

This brief report documents the numerical libraries which we have
developed. The code is written in ANSI-C, \cite{CProgLan}. This fact
makes the code very portable, especially Unix based systems should be
able to run it.

The library is purposed to be a general-purpose numerical library.
This means that there is routines for many different problems. 

The following sections describe each a library. The routines are
discussed, and general guidelines are given. The names of the
libraries are given in {\tt typewriting}, and this also gives the name
of the source and object files.

The implementation presented here should not be consider to be advanced.
It is meant to be small and simple. It contains the routines which we
have found useful in our applications, which is mainly the simulation
of chemical reactions.


\section{Integration of functions}
\label{sec:InteFunc}

The scope of the library called {\tt integr} is to evaluate definite
integration, \ie to compute

\begin{equation}
  \int_a^b f(x, \vec{\mu}) {\mathrm d}x,
\end{equation}
where $\vec{\mu}$ is a number of parameters. All the routines in the
library have the same structure. 

The function $f$ is assumed to be declared as

\begin{verbatim}
double f(double x, double *mu)
\end{verbatim}

The integration routines can be called as

\begin{verbatim}
Inte(N, a, b, mu, &f)
\end{verbatim}
where {\tt N} is the number of sunintervals which should be used, {\tt
  a}, {\tt b} and {\tt mu} correspond to $a$, $b$ and $\vec{\mu}$
respectively, and of course {\tt f} is $f$. The function returns the
value of the integral as a double-precision value.

There are three choices of integration methods available. {\tt Gauss5}
is a five-point Gaussian quadrature, {\tt Simpson} is the composite
Simpson's rule, and {\tt Trapez} is the composite trapezoid rule.


\section{Complex numbers}
\label{sec:complex}

Many numerical applications are using complex numbers, and we provide
a small implementation in form of the {\tt complex} library. 

The complex numbers are given their own type. It is called {\tt
  Complex}, and therefore a proper declaration of the complex number
$z$ is

\begin{verbatim}
Complex z;
\end{verbatim}

The real and imaginary parts of {\tt z} are address as {\tt z.re} and
{\tt z.im}, respectively.


\subsection{ComplexAssign}
\label{sec:ComplexAssign}
\begin{verbatim}
ComplexAssign(double x, double y, Complex *z)
\end{verbatim}

This routines assign the complex number $z$ a value, namely $z =
x+\imath y$.


\subsection{ComplexAdd}
\label{sec:ComplexOper}
\begin{verbatim}
ComplexAdd(Complex a, Complex b, Complex *z);
ComplexSub(Complex a, Complex b, Complex *z);
ComplexMul(Complex a, Complex b, Complex *z);
ComplexDiv(Complex a, Complex b, Complex *z);
\end{verbatim}

Depending on the routine, it computes either the sum, the difference,
the product or the quotient of the two number $a$ and $b$ and returns
the result in $z$. For {\tt ComplexDiv} the result is $\frac{a}{b}$
and for {\tt ComplexSub} is $a-b$.


\subsection{ComplexNorm}
\label{sec:ComplexNorm}
\begin{verbatim}
double ComplexNorm(Complex z);
\end{verbatim}

The function computes the norm of $z$.


\subsection{ComplexArg}
\label{sec:ComplexArg}
\begin{verbatim}
double ComplexArg(Complex z);
\end{verbatim}

the function computes the argument of $z$.


\section{Eigenvalues}
\label{sec:Eigen}
In order to calculate eigenvalues and -vectors of general real
matrices, we have implemented one routine in the library {\tt eigen}.
An eigenvalue problem is the equation

\begin{equation}
{\mathbf A}\cdot\vec{x} = \lambda\vec{x},
\label{EigenEq}
\end{equation}
where ${\mathbf A}$ is a $n\times n$ matrix, $\vec{x}$ is
$n$-dimensional vector (called eigenvector) and $\lambda$ is a
complex number (called the eigenvalues). 

The routine has a very simple interface, namely

\begin{verbatim}
void Eigen(int n, double **A, int maxiter, Complex *vals, Complex
**vecs)
\end{verbatim}

The arguments are explained in the table below.

\begin{center}
  \begin{tabular}{|l|l|}
    \hline
    Argument       & Description \\ \hline
    {\tt n}        & the dimension $n$ \\
    {\tt A}        & a real $n\times n$ matrix \\
    {\tt maxiter}  & the maximal number of iterations \\
    {\tt vals}     & the $n$ eigenvalues packed as a vector \\
    {\tt vecs}     & the $n$ eigenvectors packed as a matrix \\ \hline
  \end{tabular}
\end{center}


\section{Ordinary differential equations}
\label{sec:ODEs}
In many physical applications one wants to solve ordirary differential
equations. We have implemented a library called {\tt odesolv} which is
a good tool. The library consists of a number of methods. Each method
is implemented as a subroutine, which take one step with a given
length. Each routine returns also an error estimate.

The first group of routines take solves the equation

\begin{equation}
  \diff{\vec{x}}{t} = \vec{f}(\vec{x}),
  \label{ODEauto}
\end{equation}
\ie a autononymous ordinary differential equation of 1st order. The
routines have a common interface, namely

\begin{verbatim}
method(int n, double dt, double *x, double *xnew, double *err, 
       void (*f)(double *, double *), void (*jac)(double *))
\end{verbatim}

The arguments are summarised below.

\begin{center}
  \begin{tabular}{|l|l|}
    Argument    & Description \\ hline
    {\tt n}     & the dimension \\
    {\tt dt}    & step length \\
    {\tt x}     & $\vec{x}$ before the step \\
    {\tt xnew}  & $\vec{x}$ after the step \\
    {\tt err}   & the error estimate \\
    {\tt f}     & a pointer to $\vec{f}$ \\
    {\tt jac}   & a pointer to a function which computes the jacobian
    \\ 
    \hline
  \end{tabular}
\end{center}

The argument {\tt f} is a pointer to a function, which first argument
is $\vec{x}$ and as second argument is $\vec{f}(\vec{x})$. The
argument {\tt jac} is a pointer to a routine, which computes the
elements of the jacobian matrix, \ie

\begin{equation}
  J_{ij} = \left. \frac{\partial f_i}{\partial x_j}\right|_{\vec{x}}.
  \label{Jac}
\end{equation}

The matrix is communicated through a global variable {\tt
  jaccobi_matx} (the user must allocate it, but but define it).

Not all the routines are using the jacobian matrix, and therefore this
argument may be omitted.


\section{Vectors and matrices}
\label{sec:VecMatx}


\end{document}

