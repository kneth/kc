\hyphenation{che-mi-cal dif-fe-ren-tia-ting}
\documentstyle{article}
\author{Kenneth Geisshirt \\ \small (e-mail: kneth@\{vscht.cs, kiku.dk\}) \normalsize}
\title{Yet Another Kinetic Compiler}
\date{20 August 1992}
\begin{document}
\bibliographystyle{plain}
\maketitle
\begin{abstract}
  Yet another kinetic (pre-)compiler called {\tt kc} has been implemented. The development serves
  two goals; easy to use and easy to extend. The current version (0.~00) generates
  code to CONT, \cite{marek} and to an integrator developed at University of Copenhagen.
\end{abstract}
\section{Introduction}
A kinetic (pre-)compiler's task can be defined very simple. It transforms a chemical
kinetic model into a runnable program. In this case the runnable program is only
a subroutine to existing programs like CONT and not a complete program.

The two goals of the program development have been: 
\begin{itemize}
  \item Easy to use, i.~e.~the input is much like the 'paper model'.
  \item Easy to extend, i.~e.~new features can easily be created.
\end{itemize}

The second goal is achieved by structuring the code and by intensive use of
concrete data types, which hides the details of implementation. Heavy documentation
is also considered as a important factor in achieved the goal, because future
programmers can find how different parts are interacting.

The status of the program can be summerised to:
\begin{itemize}
  \item The precompiler is at the moment running on a HP 9000s720 under HP-UX 8.~05. 
  \item Semantic checks are not fully implemented, which means that a some 'logic' 
        errors are not detected.
  \item Code generation by the use of mass-action law and power law is working.
  \item Parameters and initial concentrations may be specified. 
\end{itemize}

In the following sections, all optional parts in grammars are placed between brackets ([ and ]).

\section{Precompilers}
There exist already quite a lot chemical precompilers 'on the market'. They are often a ad hoc solution
of a problem and not general-purpose compilers. A few of these can be mentioned.
\begin{description}
  \item[Kin.] Kin is developed at University of Copenhagen. The system generates HP-Pascal code and the 
       integrator finds the concentrations as a function of time, i.~e.~a dynamical simulation. The 
       program is able to take care of linear constraints and calls of external functions (like Arrhenuis).
       The syntax is very easy, because it looks like ordinary chemical equations. But even through it is easy
       to use, the code generator is hard to extend.
  \item[Ionic.] This is a preprocessor developed at Prague Institute of Chemical Technology. The syntax has
       to very much like Fortran-77, and the program is also generating subroutines in Fortran. The program 
       is meant to be used in one application area; diffusion controlled reactions.
  \item[CONEX.] CONEX, \cite{conex}, is a specialised front-end to CONT. It is an expert system, which assists the user in 
       making the right decision about input to CONT. It is also capable of generating the subroutine used
       by CONT. It is not flexible in the sense, that is is a front-end to one particular program.
\end{description}

The three programs above all lack of flexibility. They are hard to extend to be used to generate input to 
other programs. 

The approach in {\tt kc} is a bit different. A flexible front-end (lexical analyser, parser and symbol tables)
is developed, and back-ends (code generators) can be created when needed without changing the front-end. 
This approach has the advantage that the user will have a consistent user-interface. The future back-end
programmer is not to write new front-end. This will properly save a lot of time in the process of 
development. 

\section{Main features}
The following features are present in this version of {\tt kc}. 
\begin{itemize}
  \item Relative easy and general syntax.
  \item Code generating to CONT and Kin by the use of mass-action law and power law.
  \item Symbolic differentiating of expressions.
  \item Portability (written in ANSI-C, {\em yacc} and {\em lex}, \cite{ctools}, \cite{KR}).
\end{itemize}

The symbolic library is capable of doing ordinary algebraic manipulations like
addition and multiplication. To make the library more usable, a simplification routine
is implemented, because it is well-known that differentiating by computer often leads
to rather large expressions.

The differentiating could have been done by using automatic differentiating techniques, \cite{autodiff}. This will 
in many cases had let to more efficient subroutines, especially for very large models. 

\section{Running {\tt kc}}
To run {\tt kc} is in fact simple. The following is to be typed on the command
line.
\begin{verbatim}
kc [options] < file
\end{verbatim}
where {\tt file} is the name of the file, which contains the model. The precompiler
will respond by writing the line:
\begin{verbatim}
kc v0.00, CopyWrong by Kenneth Geisshirt, 1992.
\end{verbatim}
If there are no other messages, the compilation has been without errors and the output is written
to a file, which name depends on which code generator there is used. 

The options are in general a minus (-) and a letter. In this version three options are valid. They are 
{\em m}, {\em d} and {\em h}. Option {\em h} writes a short help text on the screen. The {\em d}
options enable the debugging mode, i.~e.~the parser will write some debug information on
the screen (to understand the output, the user have to know the theory of LALR-grammars
and their parsers). The last option ({\em m}) determines the mode. In this moment, there
are two modes available. It is the mode 1, which is using symbolic differentiating 
code generator to CONT,
and mode 4, which generates code to Kin.
The mode is written as a number after the {\em m} without
any white-spaces.

Syntax errors are only reported as {\tt syntax error}, and no indication in which line. Only a 
few other errors are detected, like defining a equilibrium constant when the reaction is a one-way
reaction.
When one syntax error has been encountered, the compiler will not continue. In finding the line
with the syntax error, the debug mode can be of great help.

\section{Syntax of input}
The input syntax is very important to every program. The input to {\tt kc} can be divided
into three parts. These three parts are {\em parameters}, {\em equations} and {\em 
constants}.
In three subsections I will examine each part, but first let's consider some
general principles.

\subsection{Numbers}
Numbers are of course of great importance for {\tt kc}. But there exists only one
form of numbers, namely floating-point numbers. The general number in other word the form
\begin{verbatim}
 xxx[.yyy[Ezzz]]
\end{verbatim}
In short, the numbers of {\tt kc} are like floating-point numbers in normal programming
languages. The main difference is, that the exponent has to have a sign ({\tt +} or {\tt -}).

\subsection{Names}
Names used in {\tt kc} follow also the common conventions of programming languages. But
there is one difference. Only letters and digits are allowed, i.\ e.\ underscore (\_),
etc.\ can not be used. And capital letters are not the same as non-capital letters, e.~g.~ 
{\tt Water} is not the same as {\tt water}. 

\subsection{Species}
Species are written the same way all through the source file. A specie consists 
in this context of a name and a charge. The charge is written in parentheses. The
syntax for a specie is:
\begin{verbatim}
  name [( charge )]
\end{verbatim}
Since radicals have to be represented this way also, the charge can either
be a number or a dot (.~). The last indicates a radical. 

The charge can of course either be positive or negative. If no charge is
given, it is assumed that the specie is neutral. Charges can be given
as floating-point numbers if it is needed.

\subsection{Parameters and inclusions}
This part of the input is optional, i.~e.~it does not have to be there. 
The part contains definitions of parameters. A parameter can be used 'later' in the
source file. An example could be the temperature. Each definition has the form
\begin{verbatim}
  name = expr
\end{verbatim}
where {\tt expr} is an expression, which can be evaluated at that point in the source
file. 

An other feature in this part is inclusion of other files. This feature is not yet 
supported.

\subsection{Equations}
The equations determined the concrete system, which is been modelled. There are two
kinds of equations; chemical and mathematical equations. The mathematical equations
have not yet been implemented, but the chemical ones are.

Every equation begins with a number. The number has to be a integer and unique for be 
equation. After the number there has to be a colon (:). 

The chemical equation has the form:
\begin{verbatim}
  number : [number] specie + ... arrow 
  [number] specie + ... ; constants ;
\end{verbatim}

The optional number before a specie is the stoichiometric coefficient. The arrow
shows which kind of reaction it is. There are three kinds of reactions. They are:
one-way, two-way and equilibrium. The notation for these are {\tt ->}, {\tt <->}
and {\tt =}. The constant-section of the equation gives the rate constants and
the possible constants for the power law. 

Rate constants and equilibrium constants are denoted by the following symbols:
{\tt k>}, {\tt k<} and {\tt K}. The first two is for reactions, resp.~left to right
and right to left. The last one is a equilibrium constant. Constants used in the power
law is denoted by {\tt c} followed by the specie in parentheses. 

The assignment for all these constants have the form:
\begin{verbatim}
  const = expr
\end{verbatim}

\subsection{Constraints, constants and initial conditions}
This part of the source file is meant to define various constants and constraints. The constants
in this part is specie-specific, i.~e.~they are associated with a specie (could be molar mass).

Only one feature is at the moment implemented; it is initial concentrations. A initial concentration
has the form:
\begin{verbatim}
[specie](0) = expr
\end{verbatim}

Not that the brackets don't mean a optional part. They show that it is a concentration. The
specie has the form already discussed. 

\section{Limits}
The program can not handle infinite large models. But the size of the 
model can be rather large. The table below shows various limits of the
current version.

\vspace{.2cm}
\begin{tabular}{|l|r|}
\hline 
 Description           &  Limit  \\ \hline \hline 
 Reactions/equations   &    250  \\ \hline
 Species + paramters   &   200   \\ \hline
 Constraints           &    50   \\ 
\hline 
\end{tabular}
\vspace{.2cm}

These values can easily be changed; they are found as macros in the
file {\tt tableman.h}.

\section{Examples}
In this section I will give some examples of input to the program.  
All the examples have the last line in common ($10 = 10$). The line has
no meaning, but every source file must contain some definitions of constants 
at the end of the file. But this part has not yet been fully implemented (only
initial concentrations).

In all examples, the rate constants are set 'randomly' i.~e.~they
have no chemical interpretation.

The first examples is a first order reaction. A compound ($A$) is transformed
into a product ($P$). The rate constant is set to $1.0$ for simplicity. The source
file is:
\begin{verbatim}
1: A -> P; k> = 1.0;
10=10;
\end{verbatim}

The next examples is more complicated. 
\begin{verbatim}
1: A -> X; k> = 1.0;
2: B + X -> Y + D; k> = 2.0;
3: 2 X + Y -> 3 X; k> = 3.0;
4: X -> E; k> = 4.0;
10=10;
\end{verbatim}

Ions are, as already mentioned, also supported in this version. An examples
of this, let's have the following red-ox reaction:
\begin{verbatim}
1: 2 Fe(+3) -> 3 Fe(+2); k> = 10.0;
10=10;
\end{verbatim}

Initial concentrations and parameters are allowed. The next example shows this.
\begin{verbatim}
T = 1000
1: A + B -> P; c(A)=1.2; c(B)=2.1; c(P)=5.0; k>=5*T;
[A](0)=0.1;
[B](0)=0.2;
\end{verbatim}

\section{Code generators}
There are at the moment two code generators implemented. The first one generates
the subroutine to CONT, while the second one generates to Kin.

In the subroutine used by both integrators have to have two quantities evaluated. They are the vector
$F$ and the matrix $G$. In this case $F$ is the rate of the reactions, while $G$ is 
the Jacobian matrix defined as $\frac{\partial F_i}{\partial x_j}$, where $\underline{x}$ is the 
concentration vector. 

The vector $F$ is generated by the mass-action law and the power-law. The code generator
is fooled, becaused the parser inserts the constants in power-expression as coefficients.
This gives a more compact code generator. For the reaction (number $i$)
\[ a A + b B + \cdots \rightarrow x X + y Y + \cdots \]
$F_i$ becomes:
\[ F_i = k [A]^a [B]^b \cdots \]
where $k$ is the rate constant. If the reaction is a two-ways reaction, $F_i$ becomes
\[ F_i = k_1 [A]^a [B]^b \cdots - k_2 [X]^x [Y]^y \cdots \]

The matrix $G$ is now 'easy' to evaluate, if all elements of $F$ is stored. 
This is the case in these two code generators, and a pseudo-code for them is:
\begin{verbatim}
for all reactions, i
  case one-way:
    F(i) = rate constant for reaction i 
    for all species, j, in reaction i
      F(i) = F(i) * [j]**S(i, j)
   case two-ways:
     F(i) = rate constant (left to right) for reaction i
     for all species, j, on left side of reaction i
       F(i) = F(i) * [j]**S(i, j)
     temp = rate constant (right to left) for reaction i
     for all species, j, on right side of reaction i
       temp = temp * [j]**(-S(i, j))
     F(i) = F(i) - temp
for all reactions, i 
  for all species in system, j
    G(i, j) = dF(i)/d[j]
\end{verbatim}

The notation {\tt S(i, j)} is the coefficient (or constant for power-law) in reaction $i$, specie $j$, {\tt [j]} is
the concentration of $j$ and {\tt **} is the exponent operator (like Fortran-77).

This is the code, which implements both code generators on a non-concrete level. The code is of course using
various library functions, which retrieve information from the data types.


\section{Future development}
In this section I will briefly discuss the future developments 
of the program could be. In the near future a few points must be considered
as natural extension of the program.
\begin{description}
 \item[Error handling.] In the current version there are only
       a few error messages. Proper error handling is important
       in the sense of user-friendliness.
 \item[Constants.] The use of specie-specific constants will, in some applications,
       be very important. The constants have to general, and not only diffusion constants.
\end{description}

As a medium-term project general expression must have 
high priority. I am thinking on general expression defining
the model, and not only by using chemical equations. There is at least two problems, which
have to be solved,
in this project. Firstly, there have to some kind of data type to handle the
expressions. Secondly, a proper code generator has to be written - maybe existing
generators can be extended in some way.

The feature of including other files must be explored more intensively. 
Questions like 'What are the files' contents?' must be answered. 

Other code generators must be added, but these extensions will
come naturally when they are needed. As mentioned in the introduction, none of the
code generators are using automatic differentiating. it could be interesting to implement
a generator using it, and then compare it with the existing ones. 

\bibliography{chaos}
\end{document}
