% Last updated: 6 July 1994
\documentstyle[12pt]{article}
\author{Kenneth Geisshirt\thanks{Email:{\tt kneth@osc.kiku.dk}}}
\title{Installation of \\ Kinetic Compiler and Integrator \\ for
  MS-DOS.}
\date{6 July 1994}
\begin{document}
\maketitle
In this brief note, I will describe how to install the MS-DOS version
of {\tt kc}. The compiler system comes with an integrator of ordinary
differential equations, and therefore it is required to have installed
a C-compiler. I recommend djgpp which is a DOS version of gcc. In the
following discussion, I will assume that djgpp is installed in a
directory called {\tt c:$\backslash$djgpp}.

Begin with making two new diectories, one for the kinetic compiler
(let us call it {\tt c:$\backslash$kci}) and one for temporary files
({\tt c:$\backslash$temp}). 

It is vital to expand the memory space used by enviroment variables,
and is done by inserting the line 

\begin{verbatim}
shell=c:\command.com /p /e:1024
\end{verbatim}

\noindent
into the file {\tt config.sys}. The enviroment space is now 1024
bytes. In the file {\tt autoexec.bat} you have to insert the line

\begin{verbatim}
call c:\djgpp\bin\setdjgpp c:/djgpp c:\djgpp
\end{verbatim}

The batch file called {\tt setdjgpp.bat} is setting up the C-compiler.
It might be a good idea to edit the lines which set the variables
GO32DIR and TMPDIR so they reflects the directory for temporary files.
Remember to expand the path, so both {\tt c:$\backslash$kci} and {\tt
  c:$\backslash$djgpp$\backslash$bin} are included.

Now copy all the files on the {\tt kc} disk into the directory called
{\tt c:$\backslash$kci}. There are a number of files; they are:

\begin{description}
\item[kci.bat] This is the main ``program''. It runs the kinetic
  compiler, the C-compiler and the simulation program. You should edit
  the directories so they reflect your own setup. If the model is in
  file {\tt foo.des}, then a simulation is done by typing {\tt kci
    foo.des}. The batch file also does some cleaning up after the
  simulation, so only the output file is left.
\item[kc] It is just the kinetic compiler.
\item[kksolver.c] Here we have the integration routines.
\end{description}

The installation of the C-compiler should be fairly simple. The files
are compressed with arj, and you begin by going to the directory where
you want to compiler ({\tt c:$\backslash$djgpp}). Now you type

\begin{verbatim}
arj x a:djgpp
\end{verbatim}

It will ask you a few questions, but just give Y as answer. The next
disks are installed with the command:

\begin{verbatim}
arj x a:djgpp.a01
\end{verbatim}

The number (a01) is increased for each disk.
\end{document}

