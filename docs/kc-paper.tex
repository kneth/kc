% Last updated: 18 Apr 1995
\documentstyle[12pt]{article}

\newcommand{\ie}{{\em i.e.\ }}
\newcommand{\eg}{{\em e.g.\ }}
\newcommand{\etc}{{\em etc.\ }}
\newcommand{\etal}{{\em et al.\ }}
\newcommand{\smbox}[1]{\mbox{{\footnotesize #1}}}
\newcommand{\diff}[2]{\frac{{\rm d}#1}{{\rm d}#2}}
\newcommand{\R}{{\rm R}}
\newcommand{\chem}[1]{\mbox{$\rm #1$}}
\author{Preben Graae S{\o}rensen, Finn Hynne and Keld Nielsen \\ Department of
  Theoretical Chemistry \\ H.C. {\O}rsted Institute \\
  Universitetsparken 5 \\ 2100 Copenhagen \\ Denmark \\
  Kenneth Geisshirt \\ Department of Life Sciences and Chemistry \\
  Roskilde University \\ P.O.\ Box 260 \\ 4000 Roskilde \\ Denmark} 
\title{Yet Another Chemical Compiler}
\date{18 April 1995}

\begin{document} 

%\bibliographystyle{unsrt}
\maketitle

\begin{abstract}
  This paper describes a new chemical compiler we have developed. The
  package consists of a program, which is able to convert a set of chemical
  reactions into a simulation program, and solvers of differential
  equations.
\end{abstract}

\section{Introduction}
Many authors have reported development of chemical compilers, see \eg
Stabler \etal \cite{Stabler78}, Carver \etal \cite{79}, Deuflhard
\etal \cite{Deuflhard}, Ipsen \etal \cite{Ipsen91}, Edelson
\cite{Edelson76}, Rasmussen \etal \cite{ChemSimul84}, Bieniasz
\cite{Bieniasz92}. But they are all 
more or less primitive compared to the one we are going to present
here. Often they have been used in a specialised subject, \eg
atmospheric chemistry, while our system is a general-purpose one.

\section{Motivation}
In general, problems in chemical kinetics are described by $n$
coupled differential equations

\begin{equation}
\label{KinODE}
  \diff{\vec{c}}{t} = \vec{f}(\vec{c}),
\end{equation}
with the initial concentrations $\vec{c}(0) = \vec{c}_0$.

If we have $n$ species and $m$ reactions, we have $n$ differential
equations, each with up to $m$ terms. It is not difficult to write
down the equations from the reactions, but the probability of making
an error is nearly 1. We human beings have one major disadvantage -
boring work is error prone, and to write down rate expressions must be
regarded as boring.

When we automatise the process, the focus is move from writting down
rate expressions to the actual simulations. In order words, we will
gain in productivity in our work.

Large kinetic models are difficult to work with. As already mentioned,
the work of transforming a kinetic model into a simulation program is
huge. In order to reduce the number of variables, many researchers apply
the quasi steady state approximation (QSSA). But the problem is that
the QSSA may not be valid; and this is not know in advance. A chemical
compiler does the use of the QSSA obsolent.

\section{The compiler}
Our chemical compiler is based on a certain input language to describe
chemical reactions. The language is formally given as an
LALR(1)\footnote{LookAhead 1 symbol, read Left to Right.}
grammar, see \eg Aho \etal \cite{Dragon}, Levine \etal \cite{YaccLex}
for a more detailed description. The advantages of using such an
abstract definition of the input format is that there exist tools
which can generate the parser automaticly. The parser is the part of a
compiler which read the input file and checks the syntax.

Furthermore the compiler is written in ANSI-C, which means that it is
easy to port to various Unix-based computer systems, and we have even
ported it to MS-DOS\footnote{It can only run on i386-based computers
or higher using a DOS-extender, \eg DJGPP or EMX.}.

An input file consists of a number of sections: constants,
reactions/equations, constraints, and initial concentrations.
Expressions found in ordinary programming languages like Fortran-77
and C are used, \ie we we below say an expression, think of it as an
expression from a program.

Constants are declared in the beginning of a file. The declaration is
simply:

\begin{verbatim}
  name = 10;
\end{verbatim}
where {\tt name} is the name of the constant, and after the equal sign
comes a (general) expression.

The reactions are written in a natural way - natural for a chemist at
least. An example is the reaction $\chem{H^+} + \chem{HO^-}
\rightarrow \chem{H_2O}$ is written as

\begin{verbatim}
 1: H(+) + HO(-) -> H2O; k>=1.0e+14;
\end{verbatim}
where {\tt k>=} represents the rate constant, and the number in the
beginning of the line is the number of the reaction. If a reaction is
written in this way, it is assumed, that it follows a rate law which
is consistent with the law of mass
action. But the user is able to supplied her own rate expression. The
Michaelis-Menten mechanism consists of three reactions, Laidler
\cite{Laidler:ChemKin}

\begin{equation}
  E + S \raisebox{-12pt}{\shortstack{$k_a$ \\ $\rightleftharpoons$ \\
$k'_a$}} (ES) \stackrel{k_b}{\rightarrow} P+E,
\end{equation}

The rate law is

\begin{equation}
  \diff{[P]}{t} = \frac{k_b[S]}{K_M+[S]}[E]_0,
\end{equation}
where $K_M = \frac{k_b+k'_a}{k_a}$. The input to our compiler could be

\begin{verbatim}
 KM = (kb+ka2)/ka1;
 1: S -> P; v>=(kb*[S])/(KM+[S])*E0;
\end{verbatim}
if we assumed that the constants are declared in the input file.

In many chemical models, there are constraints on the concentration of
some species. We have chosen that all supported constraints must have
the form

\begin{equation}
{\rm [Spec]} ={\rm  expr;}
\end{equation}
where {\tt [Spec]} denotes the concentration of {\tt Spec}, which is
contrained, and {\tt expr} is an expression, which may contain the
concentrations of other species. The constraints implemented in this
fashion is not the same as a QSSA species. While in the QSSA the
concentration becomes constant after a while, our constraints are
strictly valid at all times.

The last section of the file consists of initial concentrations. An
example is

\begin{verbatim}
[H(+)] = 1.0e-7;
\end{verbatim}

A larger example of an input file is shown below. The overall reaction
is decomposition of ozone with iron, \cite{Frank}.

\begin{verbatim}
etime = 10.0;
dtime = 0.05;
prnmode = 1;


2:  Fe(+2) + O3         -> FeO(+2) + O2;             k>=8.2e5;
3:  FeO(+2) + Fe(+2)    -> 2Fe(+3) + H2O;            k>=1.4e5;
4:  2FeO(+2)            -> 2Fe(+3) + OH(-) + HO2(-); k>=50;
5:  FeO(+2) + H2O2      -> Fe(+3) + HO2 + OH(-);     k>=1.0e4;
6:  FeO(+2) + HO2       -> Fe(+3) + O2 +OH(-);       k>=2.0e6;
11: FeO(+2)             -> Fe(+3) + OH + OH(-);      k>=1.3e-2;
13: FeO(+2) + OH        -> Fe(+3) + HO2(-);          k>=1.0e7;

[O3](0)     = 1.3e-4;
[Fe(+2)](0) = 1.1e-4;
[H2O2](0)   = 1.0e-5;
\end{verbatim}

\section{Symbolic versus numerical differentiation}
Our chemical compiler is able to do symbolic differentiation of
expression when generating the subroutines needed to do the
simulations. It is a simple matter to do this kind of operation, if
the rate expressions follows the law of mass action. The chemical
compiler described here, is able to do differentiation of any
expression. 

We are able to demonstrate that the symbolic differentiation is
better than the numerical one. Not only do we not make any
approximation errors, we also save computer time!

In the following discussion, the time of the basic operations are
denoted by $T_{\smbox{oper}}$. By basic operations, we mean addition,
subtraction, multiplication, division, and power-raising. If $f$ and
$g$ are two general expression, the time of evalutating $f \mbox{ oper }
g$ is given by

\begin{eqnarray}
    T(f\pm g) &= T(f) + T(g) + T_{\smbox{add}} \\
    T(f\cdot g) &= T(f) + T(g) + T_{\smbox{mul}} \\
    T\left(\frac{f}{g}\right) &= T(f) + T(g) + T_{\smbox{div}} \\
    T(f^g) &= T(f) + T(g) + T_{\smbox{pow}},
\end{eqnarray}
where we have assumed that time of addition is the same as for a
subtraction. The time to fetch a variable from the storage of the
computer is $T_{\smbox{var}}$, while loading a constant takes
$T_{\smbox{const}}$.

Kinetics that follows the law of mass action, can be expressed as

\begin{equation}
  f_i = \sum_{r=1}^n \nu_{ir}k_r \prod_{s=1}^m c_s^{\nu_{sr}},
\end{equation}
and the time it takes to evaluate this is

\begin{eqnarray}
  T(f_i) &=& (2n+nm)T_{\smbox{const}} + nmT_{\smbox{var}} +
(2n+nm)T_{\smbox{mul}} \nonumber \\  
& & + nmT_{\smbox{pow}} + nT_{\smbox{add}}.
\end{eqnarray}

A numerical differentiation scheme is, Kincaid \etal \cite{NumAna:KC}

\begin{equation}
  \frac{\partial f_i}{c_j} \approx \frac{f_i(c_j+h) - f_i(c_j-h)}{2h}
\equiv \frac{\delta f_i}{\delta c_j},
\end{equation}
while the element of the Jacobian matrix exactly is

\begin{equation}
  \frac{\partial f_i}{\partial c_j} = \sum_{r=1}^n \left(\nu_{ir} k_r
  \left(\prod_{s=1,s\not=j}^m c_s^{\nu_{sj}}\right)
  \nu_{ij}c_j^{\nu_{ij}-1}\right). 
\end{equation}

The time consumption for these two differentiation schemes are

\begin{eqnarray}
  T\left(\frac{\delta f_i}{\delta c_j}\right) &=& (4n + 2nm +
  4)T_{\smbox{const}} + 
  (2nm + 2)T_{\smbox{var}} \nonumber \\ 
  & & + (4n+2nm+1)T_{\smbox{mul}} + T_{\smbox{div}}
  + (2n+3)T_{\smbox{add}} + 2nmT_{\smbox{pow}}, \\
  T\left(\frac{\partial f_i}{\partial c_j}\right) &=&
  (nm+3n)T_{\smbox{const}} + nmT_{\smbox{var}} \nonumber \\
  & & + nmT_{\smbox{pow}} + (mn+3n)T_{\smbox{mul}} + nT_{\smbox{add}}.
\end{eqnarray}

As we see, the symbolic scheme is in general faster than the numerical
one. Of course it is take longer time to generate the expressions when
the chemical compiler is running, but this we will gain during the
simulation. 

\section{Numerical procedures}
\label{sec:NumProc}
The compiler has been developed in order to help to do various
numerical tasks, \ie simulations and continuations. The primary work
area for our compiler is dynamical simulations of macroscopic chemical
kinetics, and therefore we have spent such effort in developing
numerical solvers for ordinary differential equations.

\subsection{Ordinary differential equations}
\label{sec:ODEs}
The general model of chemical kinetics is

\[
  \diff{\vec{c}}{t} = \vec{f}(\vec{c}),
\]
where $\vec{c}$ is an $n$-dimensional vector and $\vec{f}$ is a
general function mapping $\R^n$ onto $\R^n$. The solution is unique if
we specify the initial conditions.

In the literature of numerical analysis one can find many numerical
schemes for solving ordinary differential equations, and we have
selected only a few. 


\subsection{Fourth order Runge-Kutta}
\label{sec:RK4}
The most simple solver is a fourth order Runge-Kutta. The method is
described \eg by Press \etal \cite[pp.\ 710--714]{NumAna:NumRec}. 

If the solution at step $n$ is $\vec{c}_n$, then the solution at $n+1$
is computed as

\begin{eqnarray*}
  \vec{k}_1 &=& h\vec{f}(\vec{c}_n), \\
  \vec{k}_2 &=& h\vec{f}(\vec{c}_n + \frac{\vec{k}_1}{2}), \\
  \vec{k}_3 &=& h\vec{f}(\vec{c}_n + \frac{\vec{k}_2}{3}), \\
  \vec{k}_4 &=& h\vec{f}(\vec{c}_n + \vec{k}_3), \\
  \vec{c}_{n+1} &=& \vec{c}_n + \frac{\vec{k}_1}{6} +
  \frac{\vec{k}_2}{3} + \frac{\vec{k}_3}{3} + \frac{\vec{k}_4}{6},
\end{eqnarray*}
where $h$ is the step length.

The method is a said previously of order 4, and it cannot solve stiff
equations.


\subsection{Calahan}
Calahan \cite{Calahan} has described a scheme for solving stiff
equations. The method can be characterised as a semi-implicit third
order Rosenbrock method.

One step is given as

\begin{eqnarray*}
  \vec{k}_{n+1} &=& h \left( {\matrix{E} - h a_1
    \matrix{J}(\vec{c}_n)} \right)^{-1} \vec{f}(\vec{c}_n), \\
  \vec{l}_{n+1} &=& h \left( {\matrix{E} - h a_1
    \matrix{J}(\vec{c}_n)} \right)^{-1} \vec{f}(\vec{c}_n + b_1
  \vec{k}_{n+1}), \\
  \vec{c}_{n+1} &=& \vec{c}_n + R_1\vec{k}_{n+1} + R_2\vec{l}_{n+1},
\end{eqnarray*}
where $h$ is the step length, $\matrix{J}$ is the Jacobian matrix of
$\vec{f}$, and $\matrix{E}$ is a unit matrix.

The constants are: $a_1 = \frac{3+\sqrt{3}}{6}$, $b_1 =
-\frac{2}{\sqrt{3}}$, $R_1 = \frac{3}{4}$, and $R_2 = \frac{1}{4}$.

\subsection{Fifth order Runge-Kutta}
\label{sec:RKFNK}

RKFNK

\subsection{Generalised Runge-Kutta}
\label{sec:GRK4T}
The Runge-Kutta scheme can be extended in order to be able to handle
stiff equations. Kaps \etal \cite{Kaps79} have deduced such a method.
The method used is modified as outlined by Press \etal \cite[pp.\
738--742]{NumRec}, Kaps \etal \cite{Kaps85}, and Hairer \etal
\cite{Hairer91}.



\subsection{Step control}
\label{sec:StepControl}




\section{Examples}
We have tested our integrator for ordinary differential equations with
many different test examples. The figures are running time in seconds.

\begin{table}[htbp]
  \begin{center}
    \leavevmode
    \begin{tabular}{lrrr}
      \hline
      Test & Calahan & RKFNC   & GRK4T  \\ \hline
      A1   & 3.01    & 1.56    & 0.83   \\
      A2   & 4.80    & 257.72  & 2.54   \\
      A3   & 1.73    & 67.22   & 0.53   \\
      A4   & 23.96   & 117.91  & 9.32   \\
      B1   & 27.61   & 4.37    & 6.06   \\
      B2   & 4.61    & 0.54    & 2.01   \\
      B3   & 6.18    & 0.69    & 1.89   \\
      B4   & 10.89   & 1.27    & 5.13   \\
      B5   & 35.54   & 6.30    & 11.59  \\
      C1   & 1.70    & 1.45    & 0.71   \\
      C2   & 1.38    & 1.20    & 0.70   \\
      C3   & 1.03    & 1.09    & 0.60   \\
      C4   & 1.63    & 1.06    & 0.71   \\
      C5   & 1.93    & 1.18    & 0.84   \\
      D1   & 4.70    & 6.16    & 2.57   \\
      D3   & 0.87    & 3.63    & 0.95   \\
      D4   & 1.61    & 71.67   & 0.75   \\
      D5   & 0.70    & 14.45   & 0.57   \\
      D6   & 0.88    &    &        \\
      Lorenz & 4.90  & 0.84  & 2.94   \\
    \hline
    \end{tabular}
  \end{center}
  \caption{Time measurements of misc.\ numerical schemes. The
    measurements were done on an i486-based PC (25 MHz) running Linux
    v1.0. The times are in seconds, and they are the total time spent
    on the integration (user + system time).}
  \label{tab:TimeNumScheme}
\end{table}

Since we are chemists, we have tested our new compiler with chemical
mechanics for different reactions. We will here give a few examples



\section{Conclusion}
We have now shown the implementation of a new chemical compiler. It is
designed in such a way, so it can be extended in the future. 

The compiler and the integrators are put in the public domain, and the
packages can be obtained from the authors. 

%\bibliography{datalogi,chemkin,fyskemi,numana}
\end{document}


